\documentclass[]{report}

\setcounter{secnumdepth}{0}
\usepackage[margin=1in]{geometry}
\usepackage{caption}
\usepackage{parskip}
\usepackage{float}
\newfloat{Equation}{H}{lop}
\usepackage[backend=biber, style=authoryear]{biblatex}
\addbibresource{./report.bib}

\title{
	The Relative Neuroplasticity of Artificial
	\linebreak
	and Biological Neural Network Configurations
}
\author{Oliver Balfour}

\date{%
	Alfred Deakin High School\\[2ex]%
	\today
}

\begin{document}
\maketitle

\begin{abstract}
	In this experiment the neuroplasticity of neural networks, both artificial and biological, was tested. Several configurations of computing machines, ranging from simple feed-forward Artificial Neural Networks (ANNs) to more complex Recurrent Neural Networks (RNNs) and Convolutional Neural Networks (CNNs), were tested, with the human brain as a baseline for performance. Also tested was a Support Vector Machine (SVM), a much older algorithm with a mathematical and not biologically-inspired approach to classification, to gauge how far the field of Machine Learning has come in the past decades both in terms of performance and neuroplasticity. Each machine was tested (with minor alterations to the input and output layers of each artificial model to account for differing input sizes) against several different classification and regression problems including but not limited to classifying handwritten digits (using the MNIST dataset), classifying the genus of irises given their petal measurements (the Iris problem), and estimating house prices. The neuroplasticity of each model was then evaluated based on the overall performance of each model in each of the problems tested.
\end{abstract}

\tableofcontents
\newpage

\section{Introduction}

The aim of this experiment is to determine how far behind in terms of neuroplasticity modern Artificial Neural Networks (ANNs) are when compared with the human brain.

It is hypothesised that:
\begin{enumerate}
	\item The human brain will marginally surpass state-of-the-art ANN configurations, and will perform well in all problem domains.
	\item The Convolutional Neural Network (CNN) will excel at digit recognition but not at estimating house prices or classifying iris genuses.
	\item The Recurrent Neural Network (RNN) will perform well on all problems.
	\item The basic feed-forward Artificial Neural Network (ANN) will perform acceptably on all problems.
	\item The Support Vector Machine (SVM) will perform well on the Iris classification problem but perform sub-par on all other problems.
\end{enumerate}

Measure the neuroplasticity of various neural network configurations by testing how well they fit to various classification and regression problems.

Problems: (all datasets available from UCI)
\begin{enumerate}
	\item MNIST digit recognition
	\item Iris classification (Fisher's dataset)
	\item Possibly sentiment analysis, but include disclaimer saying it's not going to fit well to the models due to their simplicity.
	\item Breast cancer diagnosis dataset? Could be interesting
	\item Residential building dataset (construction/sale prices etc.)
\end{enumerate}

Configurations:
\begin{enumerate}
	\item Standard feed-forward artificial neural network, 1 hidden layer
	\item Feed-forward ANN, but with 2 hidden layers
	\item Recurrent neural network, 1 hidden layer
	\item RNN, 2 hidden layers
	\item Support Vector Machine (SVM), to show the inefficiency of traditional mathematical or logical approaches compared to biologically inspired
	\item Human Brain as a control variable? Use 2/3 people as test subject
\end{enumerate}

\section{Literature Review}

https://www.ijser.org/researchpaper/A-Neuroplasticity-Brain-Plasticity-Approach-to-Use-in-Artificial-Neural-Network.pdf

https://www.sciencedirect.com/science/article/pii/S0893608015002567


\section{Background Information}

\subsection{Neural Networks}
A neural network is essentially a simplified computer abstraction of a brain, with primitive and computationally inexpensive representations of the core functions of neurons in a real life brain. Neural networks are organised into layers, usually at least three, which contain distinct neurons. The first layer takes binary or floating-point input from the host computer program, and the last returns binary output back to the host program. In feed-forward neural networks, the most common type of neural network, signals pass forward through neurons layer by layer without any form of recursion.

A neuron is typically represented by a series of numbers: an activation threshold (bias), and a series of numbers representing the strengths of connections with neurons in the previous layer. Given the outputs (activations) of the neurons in the previous layer (or given the input data from the host program in the case of the input layer), a neuron's output (activation) will equal the sum of all of previous layer's neurons activations multiplied by the connection strengths (weights), with the activation threshold (bias) of the neuron in question being subtracted. In a more traditional perceptron model, the output would be transformed such that if it is positive it is a binary 1, and if it is negative it is a binary 0; however more modern sigmoid models usually transform this value into a floating-point decimal between 0 and 1 using a learning function.

\begin{Equation}
\begin{equation}
n_a = \sum_{i=0}^{|M|}(M_{i_a} M_{i_w}) - n_b
\end{equation}
\caption{Activation of an artificial neutron}
\end{Equation}

In the above equation, \(m\) is the set of neurons from the previous layer, \(M_{i_w}\) is the weight (or axon-dendrite connection strength) of the \emph{i}th neuron, and \(M_{i_a}\) denotes the activation (output value, a floating-point number from 0 to 1 transformed by a learning function); \(n\) is the neuron whose output is being calculated and \(n_b\) its bias (activation/firing threshold.)so the above formula boils down to adding all of the connection weights multiplied by the outputs, and then subtracting the bias. A negative value means the neuron does not fire, as the threshold has not beenreached, and a positive the opposite. The output is wrapped in a function to make the output smoother and more efficient at learning, but that is an irrelevant implementation detail.

The weights and biases of each neuron and neuronal connection are optimised to serve a specific purpose through the process of training. Usually, these values will start as random numbers, and through a process known as Stochastic Gradient Descent the values are slowly optimised until the neural network model is sufficiently good at its task. (Refer to the glossary on page \pageref{itm:SGD} for more information on Stochastic Gradient Descent.) There are many applications for neural networks: A neural network may be trained to classify images (computer vision and optical character recognition), to estimate data such as house or stock prices given data about their respective markets, or even to control robots. This universiality, or neuroplasticity, is investigated in this report.

\begin{Equation}
\begin{equation}
n_a = \left\{ \begin{array}{ll}
	0 & \mbox{if } \sum_i M_{i_a} M_{i_w} \leq n_b \\
	1 & \mbox{if } \sum_i M_{i_a} M_{i_w} > n_b
\end{array}\right.
\end{equation}
\caption{Activation (output value) of a perceptron}
\end{Equation}


\section{Method}

\section{Results}

\section{Discussion}

\section{Conclusion}

\section{Appendices}

\subsection{Appendix I: Source Code}

The Python 3 source code for all of the aforementioned experiments are available online through the Git repository hosted at following URL: https://github.com/Tobsta/ScienceFair

\subsection{Appendix II: Implementation Details}

All of the deep learning models tested were implemented in the Python 3 programming language, using the Keras machine learning library.

\twocolumn
\subsection{Glossary}
\begin{description}
\item[Activation function:] See \textit{Learning Function}

\item[Artificial Neural Network (ANN):] An interconnected network of artificial (ie. simulated) neurons that receive signals (in the form of binary or floating point numbers) as input, which pass through the connections between nodes in the network toward the output layer of neurons which emits a computed result. Simple ANNs typically contain an input layer, one or more hidden layers, and an output layer.

\item[Classification problem:] A classification problem is a type of machine learning problem in which a model is trained to classify data according to a few predefined classes. Examples include Optical Character Recognition and Computer Vision problems; the most common of which is handwritten digit recognition.

\item[Convolutional Neural Network (CNN):] A neural network in which some layers are not composed of artificial perceptrons.

\item[Cost function:] The cost function estimates the accuracy (error frequency) of a model; lower is better.

\item[Dropout:] Dropout is a technique to avoid overfitting in ANNs. It works by randomly removing nodes from the neural network. (\cite{dropout14})

\item[Feed-Forward ANN:] Feed-forward ANNs are a class of ANN which have distinct layers of neurons whose signals are only fed forward. This significantly reduces the complexity of neural networks, and means that the neural network is not capable of retaining information.

\item[Hidden Layer:] Hidden layers are layers of perceptrons in an ANN that do not directly deal with input or output, but instead perform highly complex logic through their unique combinations of neural connections (weights) and activation thresholds (biases).

\item[Input Layer:] The input layer of an ANN is the layer which doesn't take input from the previous layer, but rather from a sample in a dataset specific to the problem domain the ANN is being trained for.

\item[K-Means:] K-Means is a clustering algorithm, which attempts to find groups/clusters in a dataset with a known amount of classifications by recursively estimating the centre points (centroids) of each cluster.

\item[Learning function:] A learning function is used to transform the output value of neurons to adapt the rate at which they learn.

\item[Machine Learning:] Machine learning is a subfield of computer science that deals with artificial intelligence, pattern recognition and the creation of inference engines.

\item[Neuroplasticity:] The ability of a brain or machine learning model to adapt to different problem domains without specialisation.

\item[One-Hot Encoding:] One-hot encoding is the process of converting a class ID to an array of numbers equal to the number of classes, populated with zeros, aside from the number at the index of the class ID, which is a one. To illustrate, the classes "Cat", "Dog", and "Horse" could be represented as [1,0,0], [0,1,0], and [0,0,1] with one-hot encoding. One-hot encoding is much easier for an ANN to work with, especially in an output layer.

\item[Output Layer:] The output layer of an ANN is the layer which calculates the final result. In classification problems, each neuron in the output layer will typically map to a sample class. The result of the output layer is then compared to the one-hot encoded class of the sample in the cost function to determine the error rate.

\item[Overfitting:] If a model is overfitted, it is so fine-tuned to its training dataset that it performs poorly during validation and testing.

\item[Perceptron:] A perceptron is a common, but outdated, artificial model of a neuron which has a binary output as opposed to a floating-point one. This means that they are harder to train, as a small change in inputs creates a much larger change in output.

\item[Recurrent Neural Network (RNN):] TODO

\item[Regression problem:] A regression problem is a style of problem where a model is trained to find the relationship between input and output data. One example of a regression problem is estimating the selling price of a house given data on it (number of bedrooms, size of the block, etc.)	

\item[ReLU] ReLUs (Rectified Linear Units) are a simple activation (learning) function that returns linearly unless the input is less than 0, in which case it returns 0.

\item[Sigmoid function:] The sigmoid function is a learning function that creates an S-shaped curve.

\label{itm:SGD}
\item[Stochastic Gradient Descent (SGD):] A method of minimising the error function of a neural network model (or other machine learning model) given multivariate input, in which the the input parameters are viewed as axes on an n-dimensional graph with the cost function as the Y-axis. SGD attempts to find the global minimum of the cost function without calculating it for every possible combination of parameters by rolling the ball (vector representing the weights and biases of the neural network) down the hill (this analogy only really works with 2 parameters as the X and Z axes and the cost function as the Y axis, but it can be expanded to work with n dimensions.) Local minimums are avoided by the learning rate hyper-parameter, indicative of the length of the step taken (ideally toward the global minimum) in each iteration of the SGD algorithm. The direction gradient toward the local minimum is estimated by calculating the partial derivative of the cost function with respect to each parameter, and determining which parameter, plus-minus the learning rate hyper-parameter, has the lowest result of the cost function.

\item[Softmax:] Softmax is an activation (learning) function that scales all outputs from a layer from 0 to 1, such that the sum of all outputs is 1. It is typically used for the output layer of a classification neural network, as a kind of floating-point one-hot encoding method.

\item[Support Vector Machine (SVM):] SVMs are an older method of machine learning which uses... (TODO)
\end{description}
\onecolumn

\printbibliography

\end{document}          
